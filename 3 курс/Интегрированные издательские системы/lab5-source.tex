% "Лабораторная работа № 5"

\documentclass[a4paper,12pt]{article} % тип документа

% report, book

% Русский язык

\usepackage[T2A]{fontenc}			% кодировка
\usepackage[utf8]{inputenc}			% кодировка исходного текста
\usepackage[english,russian]{babel}	% локализация и переносы


% Математика
\usepackage{amsmath,amsfonts,amssymb,amsthm,mathtools} 


\usepackage{wasysym}

% Заговолок
\author{Беленко А.В., 1 гр. 2 подгр.}
\title{Основы работы в \LaTeX{}}
\date{\today}


\begin{document} % Начало документа
\maketitle
\newpage
\section{Издательские системы}
\subsection{Издательская система TeX}
TeX --- система компьютерной вёрстки, разработанная американским профессором информатики Дональдом Кнутом в целях создания компьютерной типографии. В неё входят средства для секционирования документов, для работы с перекрёстными ссылками. В частности, благодаря этим возможностям, TeX популярен в академических кругах, особенно среди математиков и физиков.

\subsection{Дональд Кнут}
Дональд Эрвин Кнут --- американский учёный в области информатики.\\
Эмерит-профессор Стэнфордского университета, почетный доктор СПбГУ и других университетов, преподаватель и идеолог программирования, автор 19 монографий (в том числе ряда классических книг по программированию) и более 160 статей, разработчик нескольких известных программных технологий. Автор всемирно известной серии книг, посвящённой основным алгоритмам и методам вычислительной математики, а также создатель настольных издательских систем TeX и METAFONT, предназначенных для набора и вёрстки книг научно-технической тематики (в первую очередь --- физико-математических).

\subsection{Издательская система LaTeX}
LaTeX --- наиболее популярный набор макрорасширений (или макропакет) системы компьютерной вёрстки TeX, который облегчает набор сложных документов. В типографском наборе системы TeX форматируется традиционно как LaTeX.\\
Важно заметить, что ни один из макропакетов для TeX не может расширить возможностей TeX (всё, что можно сделать в LaTeX, можно сделать и в TeX без расширений), но, благодаря различным упрощениям, использование макропакетов зачастую позволяет избежать весьма изощрённого программирования.

\subsection{Лесли Лэмпорт}
Лесли Лэмпорт --- американский учёный в области информатики, первый лауреат премии Дейкстры. Разработчик LaTeX --- популярного набора макрорасширений системы компьютерной вёрстки TeX, исследователь теории распределённых систем, темпоральной логики и вопросов синхронизации процессов во взаимодействующих системах. Лауреат Премии Тьюринга 2013 года.

\section{Основные правила создания текстового документа}

\end{document} % Конец документа