% "Лабораторная работа № 6-1"

\documentclass[a4paper,12pt]{article} % тип документа

% report, book

% Русский язык

\usepackage[T2A]{fontenc}			% кодировка
\usepackage[utf8]{inputenc}			% кодировка исходного текста
\usepackage[english,russian]{babel}	% локализация и переносы


% Математика
\usepackage{amsmath,amsfonts,amssymb,amsthm,mathtools} 


\usepackage{wasysym}

\usepackage{hyperref}

%Заговолок
\author{Беленко А.В., 1 гр. 2 подгр.}
\title{Особености технологии набора технического текста в \LaTeX{}}
\date{\today}


\begin{document} % начало документа
\maketitle
\newpage
\begin{center}
\section{О \LaTeX}
\end{center}
\subsection{Для чего предназначена издательская система \LaTeX?}
Пакет позволяет автоматизировать многие задачи набора текста и подготовки статей, включая набор текста на нескольких языках, нумерацию разделов и формул, перекрёстные ссылки, размещение иллюстраций и таблиц на странице, ведение библиографии и др. Кроме базового набора существует множество пакетов расширения \LaTeX.
\subsection{В каких случаях рационально её использовать?}
\begin{flushright}
Когда нужно вводить много физических и математических формул. Так как в \LaTeX они отображаются корректно на любом устройстве.
\end{flushright}
\subsection{Какие преимущества имеет работа в этот системе?}
\begin{flushleft}
\begin{enumerate}
\item Она {\Large полностью бесплатна}.
\item Расширяема за счёт \href{https://www.ctan.org/pkg}{пакетов}.
\item Формулы отображаются корректно на всех устрйоствах.
\item Текст выглядит <<как в книге>>.
\end{enumerate}
\end{flushleft}
\subsection{Какие сложности могут возникнуть при работе в этот системе?}
Нет возможность WYSIWYG, из-за этого поначалу будет неудобно пользоваться. Также форматирование текста сложнее делать, чем в системах с WYSIWYG.
\subsection{Какие недостатки отмечают пользователи при работе с этой системой?}
\begin{itemize}
\item Отсутствие WYSIWYG
\footnote{хотя кто-то считает это плюсом, так как в этом случае не нужно отвлекаться на оформление текста},
\item \LaTeX плохо предназначен для работы с графикой.
\end{itemize}
\newpage
\section{Изменение размера шрифта}
\subsection{Комментарии и подсказки}
\scriptsize Подсказка\\
\Large Пример\\
\subsection*{Команды управления размером шрифта}
\tiny самый маленький\\
\scriptsize меньше меньшего\\
\footnotesize маньший
\small маленький\\
\normalsize обычный\\
\large большой\\
\Large больший\\
\LARGE больше большего\\
\huge очень большой\\
\Huge очень большой\\

\normalsize
Мы используем издательскую систему \TeX
\footnote{точнее \LaTeX}

Разработчиком \TeX является известный программист
\footnote{Дональд Кнут} 

\section{Форматирование текста}
\subsection{Выравнивание}
За выравнивание по центру отвечает окружение center.\\
За выравнивание по правому краю отвечает окружение flushright.\\
За выравнивание по правому краю отвечает окружение flushleft.\\
\textit{Пример}
\begin{center}
Лабораторная работа № 6
\end{center}

\begin{flushright}
\today
\end{flushright}

\begin{flushleft}
План выполнения работы
\end{flushleft}

\subsection{Создание списков}
\subsubsection{Маркированный список}
\begin{itemize}
\item MS Word
\item MS Publisher
\item Scribus
\item \LaTeX
\begin{itemize}
\item Текст
\item Таблицы
\item Графика
\item Презентация
\end{itemize}
\end{itemize}
\subsubsection{Нумерованный список}
\begin{enumerate}
\item ЛР
\item ИСР
\item ВСР
\end{enumerate}
\section{Ссылки}
Гиперссылки создаются с помощью команды href и двух обязательных аргументов
\begin{itemize}
\item самой ссылки, т. е. адреса
\item объект (слово), которое является ссылкой
\end{itemize}
\href{https://moodle.herzen.spb.ru/mod/assign/view.php?id=39031}{Пример}
\end{document} % конец документа