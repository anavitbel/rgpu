% "Инвариативная самостоятельная работа № 7"

\documentclass[a4paper,12pt]{article} % тип документа -- лсит А4 с 12 шрифтом текста

\usepackage[T2A]{fontenc}			% кодировка
\usepackage[utf8]{inputenc}			% кодировка исходного текста
\usepackage[english,russian]{babel}	% локализация и переносы

\usepackage{amsmath,amsfonts,amssymb,amsthm,mathtools} % математические символы, формулы и т. д.

\usepackage{wasysym}

\usepackage{hyperref}

\author{Беленко А.В., 1 гр. 2 подгр.} % имя автора
\title{Таблица команд} % название документа
\date{\today} % дата создани документа

\begin{document}
\maketitle
\newpage
\section*{Команды для ввода математических формул в \LaTeX}
\begin{tabular}{ l | r }
  \textbf{Назначение команды}		  & \textbf{Вид (написание) команды} \\ \hline
  включенная формула				  &\$\$ \\ \hline
  выключенная формула				  &\$\$ \$\$ \\ \hline
  формула с номером					  & begin {equation}, end{equation} \\ \hline
  log								  & log \\ \hline
  sin								  & sin \\ \hline
  lg								  & lg \\ \hline
  arcsin							  & arcsin \\ \hline
  tan								  & tan \\ \hline
  ln								  & ln \\ \hline
  cos								  & cos \\ \hline
  arctan							  & arctan \\ \hline
  arcctg							  & arcctg \\ \hline
  суммирование						  & sum \\ \hline
  предел							  & lim \\ \hline
  интеграл							  & int \\ \hline
  бесконечность						  & infty \\ \hline
  неравенство						  & neq \\ \hline
  квадратный корень					  & sqrt \\ \hline
  текст внутри формулы 				  & text \\ \hline
  скобка соответствует высоте формулы & right, left \\ \hline
  дробь								  & frac
\end{tabular}
\end{document}
© 2021 GitHub, Inc.
Terms
